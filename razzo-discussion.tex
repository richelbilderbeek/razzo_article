\section{Discussion}

We \richel{(at least I. If you do not expect this, let me know the 
reasoning why)} expected to find a bigger inference error 
when increasing the amount of co-occurring speciation events.
The reason is simple: the tree prior used assumes speciation does not
co-occur. The stronger the amount of co-occurring speciation, the stronger
the deviation from the species tree prior.

\subsection{Research}

The MBD model is a BD model that allows for co-occurring speciation,
with the same drawbacks as the BD model: it allows for an unlimited 
number of species and does not take into account that speciation takes time.
That does make the MBD model a very focussed model, only taking into 
account the essence of co-occurring speciation. The MBD model can probably
be extended to have a diversity-dependent speciation rate (as in the 
DDD model) and/or making speciation take time by adding 
an incipient species state (as in the PBD model).
The idea of this experiment, however, is to measure the impact a novel
tree prior has on the inference error. To do so, we could have picked
any other non-standard (as in, not included within BEAST2) tree prior.

\subsection{Methodological choices}

When simulating a DNA alignment, we chose to use the simplest (JC69) site
model and the simplest (strict) clock model. One could argue these models
are overly simplistic and biologically irrelevant. We argue that this
actually improves the clarity of our results for two reasons. The first
reason is a practical one: because the computational load was lowered,
we were able to perform more replicates. For a process with this high
amount of stochaticity, we think this is highly favored. The second reason
is that we think this reduces the noise of our results, without 
hurting the experiment: in our setup, it is essential
to assume the same generative site and clock model as the one that is actually 
used. Sure, we could have generated the alignment with a 
complex (GTR) site model and a complex (relaxed log-normal) clock model, then
correctly assume these were the generative site and clock model.
But that would be besides the point of the experiment: to measure the impact
of the tree prior.

For the generative tree prior, we picked the BD tree prior.


 * Choice of generative (JC69 and strict) DNA alignment 


 * Choice of generative tree prior (BD)
 * Using Nested Sampling to do model comparison
 * Using the absolute nLTT statistic as an error function

\subsection{Choices to constrain time}

 * MBD parameters
From the four MBD parameters $\lambda$, $\mu$, $\nu$ and $q$,
we investigated 1, 2, 4 and 3 different values respectively.
We chose to use only one $\lambda$, as the proportion of species
created in a co-occurent speciation event is dependent on the ratio
between $\lambda$ and a combination of $\nu$ and $q$.  

 * Crown age, and thus the number of taxa in the true species tree
 * Number of replicates
 * DNA sequence length
 * Effective sample size
 * Choice of candidate models
 
We had to constrain ourselves on the number of replicates.

We had to constrain ourselves on the number of taxa in
the true species trees.

\subsection{Future research}

This research measures the impact an MBD tree prior 
has on the inference error. 

Because the MBD tree prior has only recently been 
developed, it's biological relevance is unknown.
Would the MBD tree prior be employed on biological data of a system
in which co-occuring speciation is known to have taken a strong role,
it will be hard to conclude its performance is superior to a
rate-shift model or an early-burst model. One way to do so, is
to add the MBD other speciation models as a species tree prior
in a phylogentic tool and do a model comparision (e.g. using Nested Sampling).
Another way to do so, is measure the impact a rate-shift or early-burst model
has on the inference error in the same way as this research.
Doing so would also allow to put the quantitative results into perspective,
helping us in how to judge the error distributions. 

Expressing the inference error of an MBD process as a distribution
is a good first step in judging the impact an MBD process has on
our predictions, when compared to standard models. It is only a first
step as one still needs to subjectively judge these error distributions,
which was exactly what \verb;pirouette; was hoping to avoid. 
One way to improve our judgement, is by testing it on known species tree
priors, for example, by simulating true trees with a BD model, when assuming
a Yule tree prior.  
