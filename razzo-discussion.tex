\section{Discussion}

We \richel{(at least I. If you do not expect this, let me know the 
reasoning why)} expected to find a bigger inference error 
when increasing the amount of co-occurring speciation events.
The reason is simple: the tree prior used assumes speciation does not
co-occur. The stronger the amount of co-occurring speciation, the stronger
the deviation from the species tree prior.

\subsection{Research}

The MBD model is a BD model that allows for co-occurring speciation,
with the same drawbacks as the BD model: it allows for an unlimited 
number of species and does not take into account that speciation takes time.
That does make the MBD model a very focussed model, only taking into 
account the essence of co-occurring speciation. The MBD model can probably
be extended to have a diversity-dependent speciation rate (as in the 
DDD model) and/or making speciation take time by adding 
an incipient species state (as in the PBD model).
The idea of this experiment, however, is to measure the impact a novel
tree prior has on the inference error. To do so, we could have picked
any other non-standard (as in, not included within BEAST2) tree prior.

\subsection{Methodological choices}

When simulating a DNA alignment, we chose to use the simplest (JC69) site
model and the simplest (strict) clock model. One could argue these models
are overly simplistic and biologically irrelevant. We argue that this
actually improves the clarity of our results for two reasons. The first
reason is a practical one: because the computational load was lowered,
we were able to perform more replicates. For a process with this high
amount of stochaticity, we think this is highly favored. The second reason
is that we think this reduces the noise of our results, without 
hurting the experiment: in our setup, it is essential
to assume the same generative site and clock model as the one that is actually 
used. Sure, we could have generated the alignment with a 
complex (GTR) site model and a complex (relaxed log-normal) clock model, then
correctly assume these were the generative site and clock model.
But that would be besides the point of the experiment: to measure the impact
of the tree prior.

When simulating our phylogenies, we used an MBD model with and without
extinction. In the case in which extinction is present,
picking the BD tree prior as the generative tree prior seems justified:
it is the best-fitting standard tree prior. For the MBD model without
extinction, \richel{@Giappo: I think this is an interesting insight:} 
we could have picked the Yule tree prior. We \richel{at least I do} predicted
that the Yule tree prior will be selected as the best candidate model
more often in this scenario, which shows the sensitity of the method we used.

To compare multiple inference models, we picked the Nested Sampling
approach to directly compare marginal likelihoods. We could have used
alternative methods like \richel{forgot the term for now: where you
do an MCMC that can switch between models}. The reason we used Nested
Sampling was mostly pragmatic and we think other methods are equivalent.

We used the absolute nLTT statistic to measure the difference between
two phylogenies. Because the crown age of all phylogenies are equal,
normalization on the time has no effect at all. Also because of these
same crown ages, we could have picked many tree difference metrics, like
\richel{must check:} Pagel's delta or Pigot's delta-r. The reason we used
the nLTT statistic was mostly pragmatic. \verb;pirouette; does allow to 
use a different tree difference metric.

\subsection{Choices to constrain time}

From the four MBD parameters $\lambda$, $\mu$, $\nu$ and $q$,
we investigated 1, 2, 4 and 3 different values respectively.
We chose to use only one $\lambda$, as the proportion of species
created in a co-occurent speciation event is dependent on the ratio
between $\lambda$ and a combination of $\nu$ and $q$.  

The crown age we selected for our experiments was based on trial and error
of pilot experiments. Because the number of species in an MBD process is
expected to increase exponentially, increasing the crown age has a
profound effect on the number of taxa, with the consequence of having
runs that took days to calculate. We decided to prefer a lower number of
taxa at the gain of having more replicates. As a consequence, 
this made our research findings 
\richel{that is, I predict, a bigger error in settings with
more co-occurring speciation} more likely to happen: we know that the
choice of a Bayesian prior is most important in weaker data. 

The DNA sequence length we used was 1000 base pairs. This is a DNA
sequence length that we \richel{at least I} feel is informative enough
for the amount of taxa we simulate. Additionally (although sequencing
techniques advance at a great pace), this is -to our knowledge- 
of the same magnitude as the DNA length sequenced in the field.

Ideally, our ESSes would all be above the recommended value of 
200 \richel{I predict not all will be above 200, simply due to 
stochasticity}. Also here, we preferred to have more replicates
and a minimum median of 200 over less replicates. 

We compare our generative to model to \richel{I predict we will} 
a limited set of candidate models. Also this was for practical
reasons.

\subsection{Future research}

This research measures the impact an MBD tree prior 
has on the inference error. 

Because the MBD tree prior has only recently been 
developed, it's biological relevance is unknown.
Would the MBD tree prior be employed on biological data of a system
in which co-occuring speciation is known to have taken a strong role,
it will be hard to conclude its performance is superior to a
rate-shift model or an early-burst model. One way to do so, is
to add the MBD other speciation models as a species tree prior
in a phylogentic tool and do a model comparision (e.g. using Nested Sampling).
Another way to do so, is measure the impact a rate-shift or early-burst model
has on the inference error in the same way as this research.
Doing so would also allow to put the quantitative results into perspective,
helping us in how to judge the error distributions. 

Expressing the inference error of an MBD process as a distribution
is a good first step in judging the impact an MBD process has on
our predictions, when compared to standard models. It is only a first
step as one still needs to subjectively judge these error distributions,
which was exactly what \verb;pirouette; was hoping to avoid. 
One way to improve our judgement, is by testing it on known species tree
priors, for example, by simulating true trees with a BD model, when assuming
a Yule tree prior.  

The effect of using the absolute nLTT tree statistic to quantify the error
is unknown. A pilot experiment \richel{pirouette example 14} shows that
using the logarithmically transformed nLTT statistic gives a messier and
flatter error distribution, but this is not to be taken for granted.

We chose to use few taxa (as we preferred more replicates), but we do not
know the results would the data be more informative (that is, have more taxa).
A cool feature of \verb;pirouette; is that it can quantify when this happen;
when the prior is uninformative due to strong data.