\section{Introduction}

Modern computational techniques allow to infer phylogenies from genetic data 
such as DNA, RNA or proteins. BEAST (\cite{beast}) and its descendant 
BEAST2 (\cite{beast2}) are widely used tools to perform this task, which they 
can achieve by running a Bayesian analysis given data and tree priors.

BEAST2 gives to the user the option to set up several possible phylogenetic 
priors (e.g. substitution/clock/speciation models). However, currently 
available priors can be not suitable to analyze some specific datasets.

For this reason BEAST2 provides users with the possibility to introduce new 
tree priors, to infer phylogenies based on different assumptions on how the 
speciation process takes place.

Current phylogenetic tools assume that only a single speciation event can occur 
at any given time.
While this assumption has been proved to be useful to construct a wide 
variety of successful models (e.g \cite{Maddison2007biSSE}, 
\cite{Valente2015}, \cite{etienne2012diversity}, \cite{etienne2014estimating}), 
they disallow for environment
al changes that trigger speciations in multiple clades at a same point in time. 

We explore such case introducing the multiple birth model, currently absent 
in BEAST. The multiple birth hypothesis aims to include species pump 
mechanisms to investigate drivers and modes of such diversification processes 
whose phylogenies show an impressive amount of speciation events 
in relatively short times.

The (constant-rate) birth-death (BD) model embodies the common assumption that 
only a single speciation event can occur at any given time. The 
multiple-birth-death (MBD) model relaxes this assumption allowing, in additi
on to standard BD events, also events in which large-scale environmental 
changes lead to speciation bursts.
Such hypothesis can be useful to better understand the history of systems of 
particular interests for evolutionary biologists, such as the 
diversification of cichlid fish in the African Great Lakes (Malawi, 
Tanganyika and Victoria), where 
water level fluctuations are thought to play an important role in promoting 
diversification (\citep{verheyen1996mitochondrial}, \citep{sturmbauer2001lake}, 
\citep{janzen2016}, \citep{janzen2017}).

However, the introduction of new tree priors is not always 
desirable (\citep{bilderbeek2019pirouette}). Current BD tree priors might, 
in principle, prove to be 
good enough at detecting such events despite the lower level of complexity. 
If this is the case one should always be more keen to adopt the simplest model.

We used the R package \verb;pirouette; (\citep{pirouette}) to perform such 
test, starting on phylogenies simulated under the MBD regime using 
the \verb;mbd; package (\citep{mbd}).
From such phylogenies we measure the inference error made adopting a 
standard BD tree prior in the inference process.

With this work we aim, using such inference error distributions, to test 
whether or not it is advantageous to implement a new prior model that can 
allow the construction of trees where multiple speciations can co-occur at 
the same time.
