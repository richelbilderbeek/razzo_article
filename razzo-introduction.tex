\section{Introduction}

Modern computational techniques allow to infer phylogenies from genetic data 
such as DNA, RNA or proteins. BEAST (\cite{beast}) and its descendant 
BEAST2 (\cite{beast2}) are widely used tools to perform such task. They 
return posterior distributions of phylogenies 
and estimated parameters by running a Bayesian analysis, 
given genetic data and tree priors. 

A tree prior is a mathematical description 
of the a priori characteristics 
that we expect to be reflected in posterior phylogenies. 
The choice of a specific prior is, by definition, 
arbitrary but the consequences of such choice 
can be vet analyzing the so-obtained posteriors.

BEAST2 gives to the user the option to set up 
several possible phylogenetic 
priors (e.g. substitution/clock/speciation models). 
However, currently available priors 
might be not suitable to analyze some specific datasets.
For this reason BEAST2 provides users with the possibility 
to introduce new tree priors, 
to infer phylogenies based on different assumptions 
on how the speciation process takes place.

Current phylogenetic tools assume that 
only a single speciation event can occur at any given time.
While this assumption has been proved to be useful 
to construct a wide variety of successful 
models (e.g \cite{Maddison2007biSSE}, \cite{Valente2015}, 
\cite{etienne2012diversity}, \cite{etienne2014estimating}), 
they do not take into account the possibility 
for environmentally-driven contemporary speciations on multiple lineages.

We explore such case introducing the multiple-birth death model (MBD), 
currently absent in BEAST2. 
The multiple birth (MB) hypothesis aims to describe how large-scale environmental changes can lead phylogenies with a high abundance of speciation events in relatively short times.

\iffalse
\giovanni{This might be good for the abstract}
The (constant-rate) birth-death (BD) model embodies the common assumption 
that only a single speciation event can occur at any given time. 
The multiple-birth-death (MBD) model relaxes this assumption allowing, 
in addition to standard BD events, 
also events in which large-scale environmental changes lead 
to speciation bursts.
\fi

Such hypothesis can be useful to better understand the history of systems of particular interests for evolutionary biologists, such as the diversification of cichlid fish in the African Great Lakes (Malawi, Tanganyika and Victoria), where water level fluctuations are thought to play an important role in promoting diversification (\citep{verheyen1996mitochondrial}, \citep{sturmbauer2001lake}, \citep{janzen2016}, \citep{janzen2017}).

However, the introduction of new tree priors is not always desirable (\citep{bilderbeek2019pirouette}). Current standard birth death (BD) tree priors might, in principle, prove to be good enough at detecting such events despite their lower level of complexity.
If this is the case one should always choose to adopt the simplest model.

We used the R package \verb;pirouette; (\citep{pirouette}) to perform such test, starting on phylogenies simulated under the MBD regime using the \verb;mbd; package (\citep{mbd}).
From such phylogenies we measure the inference error made adopting a standard BD tree prior in the inference process.

With this work we aim, using such inference error distributions, to test whether or not it is advantageous to implement a new prior model that can allow the construction of trees where multiple speciations can co-occur at the same time.
