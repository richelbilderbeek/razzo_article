\begin{abstract}
% From 'How to construct a Nature summary paragraph'

% A short abstract of fewer than 200 words is acceptable.

% One or two sentences providing a basic
% introduction to the field,
% comprehensible to a scientist in any discipline.
There exist millions of species on Earth,
all originating from a common ancestor billions
of years ago.
The field of phylogenetics uses heritable material
to determine which species are closest related and what are
the mathematics that shape speciation.

% Two to three sentences of
% more detailed background, comprehensible to
% scientists in related disciplines.
In Bayesian phylogenetics, a DNA/RNA/protein alignment
is used to infer a distribution of phylogenies and parameter estimates.
To do so, we use assumptions that may be biologically unrealistic,
but may give tolerable errors.

% One sentence clearly stating the general
% problem being addressed by this particular
% study.
Contemporary inference assumes that speciation
never co-occurs.

% One sentence summarising the main
% result (with the words “here we show”
% their equivalent).
Here we show the error we make in our inference,
when nature has varying degrees of co-occurring speciation.

% Two or three sentences explaining what
% the main result reveals in direct
% comparison to what was thought to be the case
% previously, or how the main result adds to
% previous knowledge.

% One or two sentences to put the results into a
% more general context.

% Two or three sentences to provide a
% broader perspective, readily comprehensible
% to a scientist in any discipline, may be included
% in the first paragraph if the editor considers that the accessibility of the paper is significantly enhanced
% by their inclusion. Under these circumstances, the length of the paragraph can be up to 300 words.
\end{abstract}
{\bf Keywords:} computational biology, evolution, phylogenetics, Bayesian analysis, tree prior