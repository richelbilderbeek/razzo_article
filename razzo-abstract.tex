\begin{abstract}
% From 'How to construct a Nature summary paragraph'

% A short abstract of fewer than 200 words is acceptable.

% One or two sentences providing a basic
% introduction to the field,
% comprehensible to a scientist in any discipline.
There exist millions of species on Earth,
all originating from a common ancestor billions
of years ago.
The field of phylogenetics uses heritable material (e.g. DNA)
to determine the evolutionary history of species.

% Two to three sentences of
% more detailed background, comprehensible to
% scientists in related disciplines.
Starting from heritable material and explicit assumptions,
Bayesian phylogenetics allows to infer a jointly-estimated 
phylogeny and parameter estimates distribution.
One of these assumptions is the speciation model, which
mathematically describes the branching process of a phylogeny in time.
Speciation models commonly assume that speciation events are independent and disjointed. Yet, such assumption may overlook the complexity of certain processes, where environmental changes promote speciation events in multiple lineages.
% Purpose of the paper
This new layer of complexity can be captured developing a novel ad hoc speciation model. However, the introduction of a new model could be not necessary if current models are capable of describing the process, under simpler assumptions.
Here we investigate the extent of the discrepancy produced by current BEAST2's tree priors, when trying to infer back trees where speciation can co-occur.

\iffalse
This research answers the question what the impact is of using a
species tree model that assumes speciation is independent, when it
is used on phylogenies created by a tree model in which speciation can
co-occur.
\fi

% Summarising the main result
We let the impact of co-occurring speciation vary on datasets of simulated trees and show the corresponding error produced during the inference process.

% Two or three sentences explaining what
% the main result reveals in direct
% comparison to what was thought to be the case
% previously, or how the main result adds to
% previous knowledge.
We show that the extent of the inference error grows with the amount of co-occuring speciation events to establish the limits of BEAST2's standard tree priors.

% One or two sentences to put the results into a
% more general context.
These results allow phylogeneticists to judge under which circumstances the commonly used speciation model can be safely used. 

\end{abstract}

{\bf Keywords:} computational biology, evolution, phylogenetics, Bayesian analysis, tree prior,
pirouette, BEAST2, babette