\begin{abstract}
% From 'How to construct a Nature summary paragraph'

% A short abstract of fewer than 200 words is acceptable.

% One or two sentences providing a basic
% introduction to the field,
% comprehensible to a scientist in any discipline.
There exist millions of species on Earth,
all originating from a common ancestor billions
of years ago.
The field of phylogenetics uses heritable material (e.g. DNA)
to determine the evolutionary history of species.

% Two to three sentences of
% more detailed background, comprehensible to
% scientists in related disciplines.
Starting from heritable material and explicit assumptions,
Bayesian phylogenetics allows to infer a jointly-estimated 
phylogeny and parameter estimates distribution.
One of these assumptions in the speciation model, which
mathematically describes the branching process of a phylogeny in time.
The most used speciation model assumes that speciation events are
independent, where we know that certain events can trigger speciation 
events in multiple species.

% One sentence clearly stating the general
% problem being addressed by this particular
% study.
This research answers the question what the impact is of using a
species tree model that assumes speciation is independent, when it
is used on phylogenies created by a tree model in which speciation can
co-occur.

% One sentence summarising the main
% result (with the words “here we show”
% their equivalent).
Here we show the inference error made,
when nature has varying degrees of co-occurring speciation
over a wide range of parameter settings.

% Two or three sentences explaining what
% the main result reveals in direct
% comparison to what was thought to be the case
% previously, or how the main result adds to
% previous knowledge.
We show that the inference error correlates
with the amount of co-occuring speciation events,
which valudates

% One or two sentences to put the results into a
% more general context.
These results allow phylogeneticist to judge under which
circumstances the commonly used speciation model can be safely 
used. 

% Two or three sentences to provide a
% broader perspective, readily comprehensible
% to a scientist in any discipline, may be included
% in the first paragraph if the editor considers that the accessibility of the paper is significantly enhanced
% by their inclusion. Under these circumstances, the length of the paragraph can be up to 300 words.
In a bigger picture, these results showcase the use of a general and 
flexible method we used to assess the impact of using an oversimplistic tree 
prior, helping phylogeneticists to find the line between 'too simple'
and 'too complex' speciation models.

\end{abstract}

{\bf Keywords:} computational biology, evolution, phylogenetics, Bayesian analysis, tree prior,
pirouette, BEAST2, babette