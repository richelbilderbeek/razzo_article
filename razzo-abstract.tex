\begin{abstract}
% From 'How to construct a Nature summary paragraph'

% A short abstract of fewer than 200 words is acceptable.

% One or two sentences providing a basic
% introduction to the field,
% comprehensible to a scientist in any discipline.
The field of phylogenetics uses heritable material such as DNA to determine the (shared) evolutionary history of a set of species which are summarized in a phylogenetic tree.
% Two to three sentences of
% more detailed background, comprehensible to
% scientists in related disciplines.
Bayesian phylogenetic methods allow us to jointly infer probability distributions for the phylogenetic tree and parameters of the various models underlying the methods.
One of these models is the diversification model, which mathematically describes the dynamics of speciation addition and removal through time. In Bayesian analyses these are called the (species) tree priors. Tree priors commonly assume that speciation events occur independently. The Bayesian tools heavily rely on this assumption. However, under species pump dynamics, this assumption is violated. By species pump dynamics we mean the (repeated) simultaneous formation of multiple species due to environment-driven isolation which results in temporally aligned (or clustered) divergence times.
% Purpose of the paper
Current Bayesian phylogenetic tools do not contain such species pump diversification models as tree priors. This may not be a problem if currently implemented tree priors are already capable of inferring a phylogenetic tree to a satisfactory extent.
Here we investigate the extent of the error made by one such Bayesian phylogenetic tool (BEAST2) when inferring a phylogenetic tree generated by a known species pump diversification model with a standard tree prior.

% Summarising the main result
To this end we simulate our species pump model, which we call the multiple birth model because it produces multiple simultaneous speciation events, under various parameter settings, and evaluate the corresponding error produced during the inference process. We compare this error with the error made when the generating model is the same as the tree prior used in inference.

% Two or three sentences explaining what
% the main result reveals in direct
% comparison to what was thought to be the case
% previously, or how the main result adds to
% previous knowledge.
We show that the extent of the inference error increases with the number of multiple birth events, as expected.
\rampal{We need a bit more reults here}

% One or two sentences to put the results into a
% more general context.
These results allow phylogeneticists to judge under which circumstances the standard tree priors can be safely used, and under what circumstances phylogenetic inferences are unreliable. \giovanni{Not entirely sure if this is the best sentence to put here.}
\rampal{Would be nice if we gave some guidelines here}

\end{abstract}

{\bf Keywords:} computational biology, evolution, phylogenetics, Bayesian analysis, tree prior,
pirouette, BEAST2, babette